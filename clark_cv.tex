%%%%%%%%%%%%%%%%%%%%%%%%%%%%%%%%%%%%%%%%%
% Medium Length Professional CV
% LaTeX Template
% Version 2.0 (8/5/13)
%
% This template has been downloaded from:
% http://www.LaTeXTemplates.com
%
% Original author:
% Trey Hunner (http://www.treyhunner.com/)
%
% Important note:
% This template requires the resume.cls file to be in the same directory as the
% .tex file. The resume.cls file provides the resume style used for structuring the
% document.
%
%%%%%%%%%%%%%%%%%%%%%%%%%%%%%%%%%%%%%%%%%

%----------------------------------------------------------------------------------------
%	PACKAGES AND OTHER DOCUMENT CONFIGURATIONS
%----------------------------------------------------------------------------------------
\documentclass{resume} % Use the custom resume.cls style
\usepackage{hyperref}
\usepackage{etaremune}
%\pagestyle{plain}
%\pagestyle{myheadings}
\usepackage{fancyhdr}
\pagestyle{fancy}
\fancyhf{}
\lhead{\textit{Curriculum Vitae -- Brian A. Clark --} \thepage} 
\renewcommand{\headrulewidth}{0pt}
%\fancyhead{}
%\fancyfoot{}
%\fancyhead[LO,LE]{\textit{Curriculum Vitae --Brian A. Clark}}
%\fancyfoot[RO,RE]{\thepage}

\newenvironment{list1}{
  \begin{list}{\ding{113}}{%
      \setlength{\itemsep}{0in}
      \setlength{\parsep}{0in} \setlength{\parskip}{0in}
      \setlength{\topsep}{0in} \setlength{\partopsep}{0in} 
      \setlength{\leftmargin}{0.17in}}}{\end{list}}

\usepackage[left=0.75in,top=0.65in,right=0.75in,bottom=0.6in]{geometry} % Document margins
\name{Brian A. Clark} % Your name


%\address{B-1 \\ II , U.P. 208016} % Your address
%\address{123 Pleasant Lane \\ City, State 12345} % Your secondary addess (optional)
%\address{(+1)~7hh~9~8486 \\ x@ii.yzac.in} % Your phone number and email

\begin{document}

%----------------------------------------------------------------------------------------
%	CONTACT
%----------------------------------------------------------------------------------------

%191 W. Woodruff Ave  \hfill  {\it Phone:}  (614) 247-8268 \\
%Physics Research Building   \hfill {\it Email:}  clark.2668@osu.edu \\ 
%The Ohio State University   \hfill {\it Website:} \url{u.osu.edu/clark.2668} \\ 
%Columbus, OH  43210 USA   \hfill {\it OrcID / inSPIRE:} \,\,      \href{https://orcid.org/0000-0003-4089-2245}{0000-0003-4089-2245}  / \href{https://inspirehep.net/author/profile/Brian.A.Clark.1}{Brian.A.Clark.1} \\ 
\vspace{-.75cm}
\begin{rSection}{CONTACT}
\begin{tabular}{@{}p{2in}p{4in}}
191 W. Woodruff Ave             & {\it Phone:}  (614) 247-8268 \\            
Physics Research Building   & {\it Email:}  clark.2668@osu.edu 
 \\         
The Ohio State University & {\it Website:} \url{u.osu.edu/clark.2668} \\       
Columbus, OH  43210 USA  & {\it OrcID / inSPIRE:} \,\,      \href{https://orcid.org/0000-0003-4089-2245}{0000-0003-4089-2245}  / \href{https://inspirehep.net/author/profile/Brian.A.Clark.1}{Brian.A.Clark.1}\\     
\end{tabular}
\end{rSection}
%\vspace{0.25cm}

%----------------------------------------------------------------------------------------
%	RESEARCH INTERESTS
%---------------------------------------------------------------------------------------

\begin{rSection}{RESEARCH INTERESTS}
Ultra-high energy neutrino astrophysics; hardware and circuit design for RF signal conditioning; \\ mechanical/electrical systems integration; machine learning in astrophysics.
\end{rSection}
%\vspace{0.25cm}

%----------------------------------------------------------------------------------------
%	EDUCATION
%----------------------------------------------------------------------------------------

\begin{rSection}{EDUCATION}
\textbf{The Ohio State University}, Columbus, Ohio USA \hfill 2014-2019 (Expected)\\
\vspace*{-.15in}
\begin{list1}
\item[] Ph.D. in Physics--Advisor: Prof. Amy Connolly
\item [] Master of Science in Physics, June 2016
\end{list1}

\textbf{Washington University in St. Louis}, St. Louis, Missouri USA \hfill 2010-2014\\
\vspace*{-.15in}
\begin{list1}
\item[] Bachelor of Arts in Physics, {\em Cum Laude}--Advisor: Prof. Henric Krawczynski
\end{list1}
\end{rSection}
%\vspace{0.25cm}

%----------------------------------------------------------------------------------------
%	AWARDS
%----------------------------------------------------------------------------------------

\begin{rSection}{AWARDS}
National Science Foundation Graduate Research Fellowship \hfill 2016-2019 \\
APS Divison of Astrophysics Travel Award \hfill 2017 \\
OSU Graduate Enrichment Fellowship \hfill 2014-2015 \\
WUSTL Undergraudate Physics Research Fellow \hfill Summer 2011 
\end{rSection}

%----------------------------------------------------------------------------------------
%	RESEARCH EXPERIENCE
%----------------------------------------------------------------------------------------

\begin{rSection}{EXPERIENCE}
{\bf The Ohio State University}, Columbus, OH USA \hfill {\bf August 2014 - present} \\
 {\em Ph.D. Student}, Ultra-High Energy Neutrino Astrophysics
\begin{itemize}
\vspace*{.05in}
\item Active developer in the simulation, hardware, and analysis efforts in Askaryan Radio Array (ARA) collaboration to detect ultra-high energy, extra-galactic neutrinos.
\item Lead and directed the mechanical and electrical systems integration of three new neutrino detecting stations, including the management of a six person team of junior students.
\item Built and tested printed circuit boards for megahertz RF signal conditioning and power distribution, monitoring, and control.
\item Deployed to Antarctica for five weeks to lead the commissioning and calibration of five neutrino detecting stations; performed rapid, high quality on site assessment of instrument performance.
\item Developed new frequency and time-series analysis techniques to analyze radio emission from solar flares in the ARA prototype; this is the first extraterrestrial emission observed by the array.
\end{itemize}

%{\bf Washington University in St. Louis}, St. Louis, MO USA \hfill {\bf October 2012 - May 2014}\\
%{\em Undergraduate Research Associate}, X-Ray Astrophysics; Advisor Prof. Henric Krawczynksi
%\begin{list2}
%\vspace*{.05in}
%\item Participated in X-Calibur collaboration to detect x-ray and gamma-rays in the upper atmosphere.
%\item Developed Monte Carlo simulations and data analysis software to explore Stokes parameters in x-ray astronomy by using method of Bayesian confidence intervals.
%\end{list2}


\end{rSection}

%----------------------------------------------------------------------------------------
%	PUBLICATIONS
%----------------------------------------------------------------------------------------
\begin{rSection}{PUBLICATIONS}
\begin{etaremune}%[leftmargin=0.64cm]
 \item ``Observation of Reconstructable Radio Emission Coincident with an X-Class Solar Flare in the Askaryan Radio Array Prototype Station." \\
 P. Allison et. al. for the ARA Collaboration (incl. \textbf{B. A. Clark} as corresponding author) \\
 To Be Submitted to Journal of Astroparticle Physics (2018).
  \item ``Measurement of the real dielectric permittivity $\epsilon_r$ of glacial ice." \\
 P. Allison et. al. for the ARA Collaboration (incl. \textbf{B. A. Clark}) \\
 Submitted to the Journal of Glaciology (2017). \href{https://arxiv.org/abs/1712.03301}{[arXiv:1712.03301]}
   \item ``Analyzing the Data from X-ray Polarimeters with Stokes Paramters." \\
 F. Kislat,  \textbf{B. Clark}, M. Bielicke, H. Krawczynski.  \\
  \href{http://dx.doi.org/10.1016/j.astropartphys.2015.02.007}{Astroparticle Physics Vol 68 Pg 45-51 (2015).} \href{https://arxiv.org/abs/1409.6214}{[arXiv:1409.6214]} 
 \end{etaremune}
\end{rSection}

%----------------------------------------------------------------------------------------
%	SCIENTIFIC TALKS
%----------------------------------------------------------------------------------------

\begin{rSection}{SCIENTIFIC TALKS (1 Invited, 8 Contributed)}
\begin{etaremune}

\item Talk, OSU Physics Summer Seminar Series, Columbus OH. \hfill 2018/06/26 \\
{\em Ultra-High Energy Neutrino Astrophysics with Radio-Based Detectors.} 

\item Talk, OSU CCAPP Seminar, Columbus OH. \hfill 2018/05/22 \\
\href{http://ccapp.osu.edu/pastseminars.html#past}{\textit{The Askaryan Radio Array: Detector Status and Prospects for Using Directional Reconstruction in Point-Source Searches.}}

\item Talk, APS April Meeting 2018, Columbus OH. \hfill 2018/04/16 \\
\href{http://meetings.aps.org/Meeting/APR18/Session/U17.7}{{\em Directional Reconstruction as a Means of Lowering Thresholds for Point-Source Searches in the Askaryan Radio Array.}}

\item Talk, TeVPA 2017, Columbus OH. \hfill 2017/08/11 \\
\href{http://indico.cern.ch/event/615891/contributions/2648790/}{{\em The Askaryan Radio Array: Current Status and Future Plans.} }

\item Talk, APS April Meeting 2017, Columbus OH. \hfill 2017/01/31 \\
\href{http://meetings.aps.org/Meeting/APR17/Session/Y3.2}{{\em Observation of Reconstructable Radio Waveforms from Solar Flares with Askaryan Radio Array.}}

\item Invited Talk, College of Wooster Physics Department Colloquium, Wooster OH. \hfill 2016/10/04 \\
{\em Ultra-High Energy Neutrino Astrophysics with Radio Detectors.}

\item Talk, Computing in High Energy Astropart. Phys. Research 2016, Columbus OH. \hfill 2016/05/26 \\
\href{http://ccapp.osu.edu/workshops/CHEAPR2016/workshop.html}{\em Machine Learning Prospects in Trigger Thresholds for High Energy Radio Neutrino Astronomy.}

\item Talk, OSU Physics Summer Seminar Series, Columbus OH. \hfill 2016/04/23 \\
{\em Trigger Thresholds in High Energy Neutrino Astronomy.} 

\item Talk, Ohio Section of the APS Spring 2016 Meeting, Dayton OH. \hfill 2016/04/09 \\
\href{http://meetings.aps.org/Meeting/OSS16/Session/D3.6}{\em Ultra-High Energy Neutrino Astrophysics with the Askaryan Radio Array (ARA).} 
 \end{etaremune}
\end{rSection}
\vspace{-0.10cm}

%----------------------------------------------------------------------------------------
%	PUBLICATIONS
%----------------------------------------------------------------------------------------
\begin{rSection}{TEACHING}
TA Training Facilitator, University Center for the Advancement of Teaching, OSU \hfill {August 2016}\\
Teaching Assistant, ``Astronomy 1143: Stars, Galaxies, and Cosmology, OSU \hfill {Spring 2016}\\
Teaching Assistant, ``Physics 1251: E\&M, Optics, and Quantum Mechanics", OSU \hfill {Fall 2015}\\
\end{rSection}
\vspace{-0.35cm}
%----------------------------------------------------------------------------------------
%	OUTREACH
%----------------------------------------------------------------------------------------
\begin{rSection}{OUTREACH AND SERVCE}
Physics Climate and Diversity Committee, OSU \hfill January 2017-present\\
Coordinator for \href{u.osu.edu/aspire}{ASPIRE} Workshop for High School Girls, OSU \hfill July 2015-present\\
Volunteer Judge, Ohio State Science Day \hfill 2015-present\\
Talk, Columbus Science Pub \hfill  May 2018\\
Talk, The Wellington School, Columbus, OH \hfill April 2018\\
Representative, Physics Graduate Student Council, OSU \hfill October 2014-May 2017 \\
%Volunteer Judge, OSU Denman Undergraduate Research Forum \hfill 2016\\
\end{rSection}
\vspace{-0.30cm}

\newpage
\begin{rSection}{MENTORSHIP}
{\bf Graduate Students:} Keith McBride, Jorge Torres-Espinosa\\
{\bf Undergraduate Students:}  Ian Best, Suren Gourapura, Hanna Hassan, Spoorthi Nagasmudram, Jude Rajasekera, Lucas Smith, Jason Torok \\
{\bf High School Students:} Addison Hartman, Natalie Keyes\\
\end{rSection}
\end{document}
