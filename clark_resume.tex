%%%%%%%%%%%%%%%%%%%%%%%%%%%%%%%%%%%%%%%%%
% Medium Length Professional CV
% LaTeX Template
% Version 2.0 (8/5/13)
%
% This template has been downloaded from:
% http://www.LaTeXTemplates.com
%
% Original author:
% Trey Hunner (http://www.treyhunner.com/)
%
% Important note:
% This template requires the resume.cls file to be in the same directory as the
% .tex file. The resume.cls file provides the resume style used for structuring the
% document.
%
%%%%%%%%%%%%%%%%%%%%%%%%%%%%%%%%%%%%%%%%%

%----------------------------------------------------------------------------------------
%	PACKAGES AND OTHER DOCUMENT CONFIGURATIONS
%----------------------------------------------------------------------------------------
\documentclass{resume} % Use the custom resume.cls style
\usepackage{hyperref}
\usepackage{etaremune}
\usepackage{tabularx}
%\pagestyle{plain}
%\pagestyle{myheadings}
\usepackage{fancyhdr}
\pagestyle{fancy}
\fancyhf{}
\renewcommand{\headrulewidth}{0pt}
%\fancyhead{}
%\fancyfoot{}
%\fancyhead[LO,LE]{\textit{Curriculum Vitae --Brian A. Clark}}
%\fancyfoot[RO,RE]{\thepage}

\newenvironment{list1}{
  \begin{list}{\ding{113}}{%
      \setlength{\itemsep}{0in}
      \setlength{\parsep}{0in} \setlength{\parskip}{0in}
      \setlength{\topsep}{0in} \setlength{\partopsep}{0in} 
      \setlength{\leftmargin}{0.17in}}}{\end{list}}

\usepackage[left=0.75in,top=0.65in,right=0.75in,bottom=0.6in]{geometry} % Document margins
\name{Brian A. Clark} % Your name


%\address{B-1 \\ II , U.P. 208016} % Your address
%\address{123 Pleasant Lane \\ City, State 12345} % Your secondary addess (optional)
%\address{(+1)~7hh~9~8486 \\ x@ii.yzac.in} % Your phone number and email

\begin{document}

%----------------------------------------------------------------------------------------
%	CONTACT
%----------------------------------------------------------------------------------------

%191 W. Woodruff Ave  \hfill  {\it Phone:}  (614) 247-8268 \\
%Physics Research Building   \hfill {\it Email:}  clark.2668@osu.edu \\ 
%The Ohio State University   \hfill {\it Website:} \url{u.osu.edu/clark.2668} \\ 
%Columbus, OH  43210 USA   \hfill {\it OrcID / inSPIRE:} \,\,      \href{https://orcid.org/0000-0003-4089-2245}{0000-0003-4089-2245}  / \href{https://inspirehep.net/author/profile/Brian.A.Clark.1}{Brian.A.Clark.1} \\ 
\vspace{-1cm}
%\begin{rSection}{CONTACT}
\rule{\textwidth}{0.1cm} \\ \\
\begin{tabular}{@{}p{2in}p{4in}}
1620 N. High Street, Apt 422             & {\it Phone:}  (314) 365-2605 \\            
Columbus, OH 43201  & {\it Email:}  baclark7777@gmail.com \\   
\end{tabular}
%\end{rSection}
%\vspace{0.25cm}

%----------------------------------------------------------------------------------------
%	OBJECTIVE
%---------------------------------------------------------------------------------------

\begin{rSection}{OBJECTIVE}
Leverage my 5+ years of experience building, simulating, and analyzing data from radio-frequency based scientific instruments to solve critical electrophysics problems in aerospace applications.
\end{rSection}
%\vspace{-0.1cm}

%----------------------------------------------------------------------------------------
%	EDUCATION
%----------------------------------------------------------------------------------------


\begin{rSection}{EDUCATION}
\textbf{Ph.D in Physics, The Ohio State University}, Columbus, OH \hfill 2019 (Expected)\\
\vspace*{-.2in}

\textbf{M.S. in Physics, The Ohio State University}, Columbus, OH \hfill 2016 \\
\vspace*{-.2in}

\textbf{B.A. in Physics, \textit{cum laude}, Washington University}, St. Louis, MO \hfill 2014\\
\vspace*{-.2in}
\end{rSection}
%\vspace{-0.1cm}

%----------------------------------------------------------------------------------------
%	RELEVANT SKILLS
%----------------------------------------------------------------------------------------
\begin{rSection}{SKILLS and AWARDS}

\begin{tabular}{@{}l l l@{}}
Programming/Software & & C++/C, Python, Bash/Shell, R, Matlab \\ 
Mechanical/Electrical & & Surface mount soldering, power distribution, RF signal conditioning  \\ 
Technical & & Grant Writing, Statistics, Calculus, Linear Algebra, Electromagnetism\\
\end{tabular}

National Science Foundation Graduate Research Fellowship \hfill 2016-2019 \\
OSU Graduate Enrichment Fellowship \hfill 2014-2015
\end{rSection}

%----------------------------------------------------------------------------------------
%	RESEARCH EXPERIENCE
%----------------------------------------------------------------------------------------

\begin{rSection}{EXPERIENCE}
{\bf Ph.D. Research Fellow, The Ohio State University}, Columbus, OH \hfill {\bf Aug 2014 - present}\\
{\em Research Focus: Ultra-High Energy Neutrino Astrophysics with Radio-Based Detectors}
\begin{itemize}
\vspace*{.05in}
\item Developed novel frequency and time-series analysis techniques for MHz-GHz RF data, and applied those techniques in analyses of radio emission from solar flares (published in 2018 \href{https://arxiv.org/abs/1807.03335}{[arXiv:1807.03335]})
\item Implemented filtering and phasing techniques to remove human-made noise from RF data, and applied those techniques in a low-SNR search for neutrinos in an large 80TB data set. 
\item Lead the mechanical and electrical systems integration of the power and signal conditioning subsystems of three neutrino detecting stations, including the management of a three person team of junior students. Resulted in a 80\% reduction to the subsystem cost.
\item Deployed to Antarctica for five weeks to lead the commissioning and calibration of five neutrino detecting stations; performed rapid on site assessment of instrument performance.
\item Created and managed automated quality control software for a large Monte Carlo simulation package, supporting a team of several dozen international scientists.
\end{itemize}

{\bf Teaching Assistant, The Ohio State University}, Columbus, OH \hfill {\bf Aug 2015 - Aug 2016}
\begin{itemize}
\item Aug 2016: Facilitated two-day ``introduction to teaching and learning`` workshop for 30 first-time Teaching Assistants; guided development of teaching identities and planning for classroom success.
\item Spring 2016: Served as teaching assistant for 80 student introductory survey course; designed evaluation instruments and moderated online student forum.
\item Fall 2015: Guided student learning in the recitation and laboratory context; facilitated quantitative laboratory experiments including team-based problem solving exercises.

\end{itemize}

\end{rSection}

\end{document}
