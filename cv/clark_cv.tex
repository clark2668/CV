%%%%%%%%%%%%%%%%%%%%%%%%%%%%%%%%%%%%%%%%%
% Medium Length Professional CV
% LaTeX Template
% Version 2.0 (8/5/13)
%
% This template has been downloaded from:
% http://www.LaTeXTemplates.com
%
% Original author:
% Trey Hunner (http://www.treyhunner.com/)
%
% Important note:
% This template requires the resume.cls file to be in the same directory as the
% .tex file. The resume.cls file provides the resume style used for structuring the
% document.
%
%%%%%%%%%%%%%%%%%%%%%%%%%%%%%%%%%%%%%%%%%

%----------------------------------------------------------------------------------------
%	PACKAGES AND OTHER DOCUMENT CONFIGURATIONS
%----------------------------------------------------------------------------------------
\documentclass{resume} % Use the custom resume.cls style
\usepackage{hyperref}
\usepackage{etaremune}
\usepackage{tabularx}
\usepackage{setspace}
\usepackage[none]{hyphenat}
%\pagestyle{plain}
%\pagestyle{myheadings}
\usepackage{fancyhdr}
\pagestyle{fancy}
\fancyhf{}
\lhead{\vspace{0.3cm}\textit{Curriculum Vitae -- Brian A. Clark --} \thepage} 
\renewcommand{\headrulewidth}{0pt}
%\fancyhead{}
%\fancyfoot{}
%\fancyhead[LO,LE]{\textit{Curriculum Vitae --Brian A. Clark}}
%\fancyfoot[RO,RE]{\thepage}

\newenvironment{list1}{
  \begin{list}{\ding{113}}{%
      \setlength{\itemsep}{0in}
      \setlength{\parsep}{0in} \setlength{\parskip}{0in}
      \setlength{\topsep}{0in} \setlength{\partopsep}{0in} 
      \setlength{\leftmargin}{0.17in}}}{\end{list}}

\usepackage[left=0.75in,top=0.65in,right=0.75in,bottom=0.6in]{geometry} % Document margins
\name{Brian A. Clark} % Your name

\usepackage{color,hyperref, url}
\definecolor{darkblue}{rgb}{0.0,0.0,0.75}
%\hypersetup{colorlinks,breaklinks,
 % linkcolor=darkblue,urlcolor=darkblue,
 % anchorcolor=darkblue,citecolor=darkblue}
\usepackage[backend=bibtex,
           bibstyle=numeric,
           sorting=none,
           sortcites=true,
           natbib=true,
           defernumbers=true,
           maxbibnames=99,
           giveninits=true,
           uniquename=false]{biblatex}
\DeclareFieldFormat{labelnumberwidth}{{#1\adddot}}

% do some fancy business to get the 

% Count total number of entries in each refsection
\AtDataInput{%
  \csnumgdef{entrycount:\therefsection}{%
    \csuse{entrycount:\therefsection}+1}}

% Print the labelnumber as the total number of entries in the
% current refsection, minus the actual labelnumber, plus one
\DeclareFieldFormat{labelnumber}{\mkbibdesc{#1}}    
\newrobustcmd*{\mkbibdesc}[1]{%
  \number\numexpr\csuse{entrycount:\therefsection}+1-#1\relax}

\addbibresource{../publist/papers.bib}


\begin{document}

%----------------------------------------------------------------------------------------
%	CONTACT
%----------------------------------------------------------------------------------------

\vspace{-1cm}
%\begin{rSection}{CONTACT}
\rule{\textwidth}{0.1cm} \\ \\
\begin{tabular}{@{}p{2in}p{4in}}
567 Wilson Rd Room 3243             & {\it Phone:}  (517) 884-5712 \\            
Biomed Phys Sci Building   & {\it Email:}  baclark@msu.edu
 \\         
Michigan State University & {\it Website:} \url{https://icecube.wisc.edu/~brianclark/} \\       
East Lansing, MI  48824 USA  & {\it OrcID / inSPIRE:} \,\,      \href{https://orcid.org/0000-0003-4089-2245}{0000-0003-4089-2245}  / \href{https://inspirehep.net/author/profile/Brian.A.Clark.1}{Brian.A.Clark.1}\\     
\end{tabular}
%\end{rSection}
%\vspace{0.25cm}

%----------------------------------------------------------------------------------------
%	RESEARCH INTERESTS
%---------------------------------------------------------------------------------------

\begin{rSection}{RESEARCH PROFILE}
NSF Astronomy and Astrophysics Postdoctoral Fellow working in experimental particle astrophysics on the Askaryan Radio Array (ARA) and IceCube experiments. Interested in high-energy neutrino astronomy, specifically the construction, simulation, and data analysis of neutrino telescopes.
\end{rSection}
%\vspace{-0.1cm}

%----------------------------------------------------------------------------------------
%	EDUCATION
%----------------------------------------------------------------------------------------

\begin{rSection}{EDUCATION}
\textbf{Ph.D. in Physics, The Ohio State University}, Columbus, Ohio USA \hfill 2014-2019\\
\vspace*{-.15in}
\begin{list1}
\item[]Thesis: \textit{Optimization of a Search for Ultra-High Energy Neutrinos in \\ Four Years of Data of ARA Station 2} 
\item[]Advisor: Prof. Amy Connolly
\end{list1}

\textbf{M.S. in Physics, The Ohio State University}, Columbus, Ohio USA \hfill 2014-2016

\textbf{B.A. in Physics, Washington University in St. Louis}, St. Louis, Missouri USA \hfill 2010-2014\\
\vspace*{-.15in}
\begin{list1}
\item[]Thesis: \textit{Calculating Bayesian Confidence Intervals \\ with Stokes Parameters for use in Hard X-Ray Polarimetry} 
\item[] \textit{Cum Laude}, Advisor: Prof. Henric Krawczynski
\end{list1}
\end{rSection}
%\vspace{-0.1cm}

%----------------------------------------------------------------------------------------
%	AWARDS
%----------------------------------------------------------------------------------------

\begin{rSection}{FELLOWSHIPS and AWARDS}
%\begin{doublespacing}
NSF Astronomy and Astrophysics Postdoctoral Fellowship \hfill 2019-2022 \\
Japanese Society for the Promotion of Science Postdoctoral Fellowship \hfill 2022 \\
NSF Graduate Research Fellowship \hfill 2016-2019 \\
OSU Graduate Enrichment Fellowship \hfill 2014-2015 \\
WUSTL Undergraduate Physics Research Fellow \hfill 2011 \\
\vspace{-0.3cm}

APS Division of Astrophysics Travel Award \hfill 2017, 2019 \\
Bunny and Thomas Clark Graduate Scholarship Honorable Mention \hfill 2019

% THIS LINE BREAK IS IMPORTANT

%\end{doublespacing}
\end{rSection}
%\vspace{-0.1cm}

%----------------------------------------------------------------------------------------
%	PUBLICATIONS
%----------------------------------------------------------------------------------------
\begin{rSection}{SELECTED PUBLICATIONS}
It is the policy of the ARA and IceCube collaborations that authors be listed in alphabetical order. \\  A full publication list is available at the end of the CV and at my \href{https://orcid.org/0000-0003-4089-2245}{ORCID}  or  \href{https://inspirehep.net/author/profile/Brian.A.Clark.1}{INSPIRE-HEP} pages.
%Publications listed here are primary author publications or publications with specific contributions. I am a co-authors  a co-author on all ARA papers since 2017 and all IceCube papers since 2020. 

\begin{etaremune}%[leftmargin=0.64cm]

  \item ``toise: a framework to describe the performance of high-energy neutrino detectors" \\
 J. van Santen, \textbf{B. A. Clark}, R. Halliday, S. Hallman, A. Nelles  \\ Submitted to JINST.  \href{https://arxiv.org/abs/2202.11120}{[arXiv:2202.11120]}


  \item ``Simulation and Sensitivity for a phased IceCube-Gen2 deployment" \\
 \textbf{B. A. Clark}, R. Halliday for the IceCube-Gen2 Collaboration \\ \href{https://doi.org/10.22323/1.395.1186}{Proc. 37th International Cosmic Ray Conference PoS (ICRC2021)1186.}  \href{https://arxiv.org/abs/2107.08500}{[arXiv:2107.08500]}
 
%\textit{Contribution: estimated the performance of the optical component of IceCube-Gen2, including producing and analyzing the simulation results; led the writing process.}

  \item ``Sensitivity Studies for the IceCube-Gen2 radio array" \\
 S. Hallmann, \textbf{B. A. Clark}, C. Glaser, D. Smith for the IceCube-Gen2 Collaboration \\ \href{https://doi.org/10.22323/1.395.1183}{Proc. 37th International Cosmic Ray Conference PoS (ICRC2021)1183.}  \href{https://arxiv.org/abs/2107.08910}{[arXiv:2107.08910]}

%\textit{Contribution: produced and analyzed the simulations to estimate the diffuse and point-source sensitivity of the radio component of IceCube-Gen2; co-led the writing process.}

\item ``The IceCube-Gen2 Neutrino Observatory" \\
\textbf{B. A. Clark} for the IceCube-Gen2 Collaboration \\ \href{https://doi.org/10.22323/1.395.1183}{Proc. 9th Very Large Volume Neutrino Telescope Workshop (VLVnT-2021).}  \href{https://arxiv.org/abs/2108.05292}{[arXiv:2108.05292]}

%\textit{Contribution: produced and analyzed the simulations to estimate the diffuse and point source sensitivity of the radio component of IceCube-Gen2; co-led the writing process.}

  \item ``Design and Sensitivity of the Radio Neutrino Observatory in Greenland (RNO-G)" \\
 J.A. Aguilar {\it et al.} for the RNO-G Collaboration (incl. \textbf{B. A. Clark})\\ \href{https://doi.org/10.1088/1748-0221/16/03/P03025}{JINST 16 (2021) 03, P03025.}  \href{https://arxiv.org/abs/2010.12279}{[arXiv:2010.12279]}

  \item ``Constraints on the diffuse flux of ultrahigh energy neutrinos from four years of Askaryan Radio Array Data in two stations" \\
 P. Allison {\it et al.} for the ARA Collaboration (incl. \textbf{B. A. Clark} as corresponding author)\\ \href{https://doi.org/10.1103/PhysRevD.102.043021}{Physical Review D 102, 043021 (2020).}  \href{https://arxiv.org/abs/1912.00987}{[arXiv:1912.00987]}

%\textit{Contribution: led the analysis from beginning to end, including designing event selections and studying systematic uncertainties. Analysis sets the strongest limit by an in situ neutrino telescope. Led the paper writing and served as the corresponding author. }

  \item ``Long-baseline horizontal radio-frequency transmission through polar ice" \\
 P. Allison {\it et al.} for the ARA Collaboration (incl. \textbf{B. A. Clark})\\    \href{https://iopscience.iop.org/article/10.1088/1475-7516/2020/12/009}{JCAP Vol 2020 No 12 Pg 009.} \href{https://arxiv.org/abs/1908.10689}{[arXiv:1908.10689]}

  \item ``NuRadioMC: Simulating the radio emission of neutrinos from interaction to detector" \\
 C. Glaser {\it et al.} (incl. \textbf{B. A. Clark})\\     \href{https://doi.org/10.1140/epjc/s10052-020-7612-8}{European Physical Journal C 80, 77 (2020).} \href{https://arxiv.org/abs/1906.01670}{[arXiv:1906.01670]}
 
% \textit{Contribution: core developer of the signal propagation (ray tracing) code, which has world's-fastest performance; led the writing of sections related to signal propagation.}

  \item ``Design and Performance of an Interferometric Trigger Array for Radio Detection of High-Energy Neutrinos" \\
 P. Allison {\it et al.} for the ARA Collaboration (incl. \textbf{B. A. Clark}) \\    \href{https://doi.org/10.1016/j.nima.2019.01.067}{Nuclear Instruments and Methods A 930, 112-125 (2019).}  \href{https://arxiv.org/abs/1809.04573}{[arXiv:1809.04573]}
 
% \textit{Contribution: deployed to Antarctica to commission the instrument and demonstrate performance; developed the co-located ARA data acquisition system which provides power and communications.}

 \item ``Observation of Reconstructable Radio Emission Coincident with an X-Class Solar Flare in the Askaryan Radio Array Prototype Station." \\
P. Allison {\it et al.} for the ARA Collaboration (incl. \textbf{B. A. Clark} as corresponding author) \\
 Submitted to Astroparticle Physics. \href{https://arxiv.org/abs/1807.03335}{[arXiv:1807.03335]}
 
%  \textit{Contribution: led the study of the events, demonstrating for the first time a radio neutrino detector can reconstruct events to an extraterrestrial object on an event-by-event basis. Led the paper writing process and serving as corresponding author.}

  \item ``Measurement of the real dielectric permittivity $\epsilon_r$ of glacial ice." \\
 P. Allison {\it et al.} for the ARA Collaboration (incl. \textbf{B. A. Clark}) \\
%  \href{https://doi.org/10.1016/j.astropartphys.2019.01.004}{Astroparticle Physics Vol 108 Pg 63-73 (2019).} \href{https://arxiv.org/abs/1712.03301}{[arXiv:1712.03301]} 

   \item ``Analyzing the Data from X-ray Polarimeters with Stokes Parameters." \\
 F. Kislat,  \textbf{B. Clark}, M. Bielicke, H. Krawczynski.  \\
  \href{http://dx.doi.org/10.1016/j.astropartphys.2015.02.007}{Astroparticle Physics 68, 45-51 (2015).} \href{https://arxiv.org/abs/1409.6214}{[arXiv:1409.6214]} 
  
%   \textit{Contribution: developed the core technique, and led the writing of the initial paper draft.}


 \end{etaremune}
 
\end{rSection}

%----------------------------------------------------------------------------------------
%	RESEARCH EXPERIENCE
%----------------------------------------------------------------------------------------

\begin{rSection}{RESEARCH EXPERIENCE}
{\bf Michigan State University}, East Lansing, MI USA \hfill {\bf August 2019 - present} \\
{\em Postdoctoral Fellow}%, High Energy Neutrino Astrophysics
\begin{itemize}
\vspace*{-.05in}
\item Led the latest ARA search for UHE neutrinos, producing the strongest limit by an in-ice radio detector and observing the first UHECR candidate events in ARA.
\item Optimized the geometry of the IceCube-Gen2 optical and radio arrays for maximum physics reach---included performing and evaluating simulations from beginning to end.
\item Evaluated the reconstruction capability of ARA stations for the first time, including estimating the resolution on signal polarization and deposited shower energy.
\item Developed Monte Carlo simulation tools for use in the neutrino community, including a world's-fastest radio signal propagation code and maintenance of a high-energy shower simulation tool.
\item Member of the ARA, IceCube, and RNO-G collaborations.  Served as analysis convener for ARA ($\sim60$ person collaboration). Led of the MSU IceCube Machine Learning Subgroup (3 graduate and 6 undergraduate students). Coordinated ARA analysis ``bootcamp" ($\sim25$ attendees).
\end{itemize}
\end{rSection}
\clearpage

\begin{rSection}{RESEARCH EXPERIENCE (cont.)}

{\bf The Ohio State University}, Columbus, OH USA \hfill {\bf August 2014 - July 2019} \\
{\em Ph.D. Student}%, Ultra-High Energy Neutrino Astrophysics
\begin{itemize}
\vspace*{-.05in}
\item Developed frequency and time-series analysis techniques to analyze radio emission from solar flares in the ARA prototype station; this is the first extraterrestrial emission observed by the array.
\item Implemented filtering techniques to remove human-made noise from ARA data, and utilized them in a search for a diffuse flux of ultra-high energy neutrinos.
\item Developed, built, and tested printed circuit boards for RF signal conditioning and power distribution, improving instrument dynamic range and operability in harsh environments.
\item Led and directed the mechanical and electrical systems integration of three new neutrino detecting stations, including the management of a three person team of junior students.
\item Deployed to Antarctica for five weeks to lead the commissioning and calibration of five neutrino detecting stations; performed rapid on site assessment of instrument performance.
\item Member of the ARA collaboration; led ARA operations coordination for one year.
\end{itemize}

{\bf Washington University in St. Louis}, St. Louis, MO USA \hfill {\bf October 2012 - May 2014}\\
{\em Undergraduate Research Associate}%, X-Ray Astrophysics
\begin{itemize}
\vspace*{-.05in}
\item Member of the X-Calibur collaboration to detect x-rays in the upper atmosphere, including fabrication of CCDs in a cleanroon environment.
\item Wrote Monte Carlo simulations to explore Stokes parameters in x-ray astronomy by using methods of Bayesian confidence intervals.
\end{itemize}
\end{rSection}

%----------------------------------------------------------------------------------------
%	TEACHING
%----------------------------------------------------------------------------------------
\begin{rSection}{TEACHING EXPERIENCE}
{\bf The Ohio State University}, Columbus, OH 

{\em TA Training Facilitator, University Center for the Advancement Teaching} \hfill {\bf August 2016}
\begin{itemize}
\vspace*{-.05in}
\item Facilitated two-day ``introduction to teaching and learning`` workshop for 30 first-time Teaching Assistants across the University's 40 STEM science programs.
\item Built confidence in new TAs, guided development of teaching identities, addressed diversity in the classroom, and aided participant planning for long-term classroom success.
\end{itemize}

{\em Teaching Assistant--``Astronomy 1143: Stars, Galaxies, and Cosmology''} \hfill {\bf Spring 2016}
\begin{itemize}
\vspace*{-.05in}
\item Aided student learning by teaching review sessions and lecturing when lead faculty was absent for 80 student introductory survey course, open to students across the university
\item Moderated online forum, in collaboration with lead faculty, for students to exchange questions and clarify concepts.
\end{itemize}
{\em Teaching Assistant--``Physics 1251: E\&M, Optics, and Quantum Mechanics''} \hfill {\bf Fall 2015}
\begin{itemize}
\vspace*{-.05in}
\item Guided student learning in the recitation and laboratory context for four contact hours per week.
\item Facilitated quantitative laboratory experiments including team-based problem solving exercises.
\item Designed rubrics for fair, efficient, and consistent grading of quiz and examination instruments.
\end{itemize}
\end{rSection}
\newpage
%----------------------------------------------------------------------------------------
%	SCIENTIFIC TALKS
%----------------------------------------------------------------------------------------

\begin{rSection}{SCIENTIFIC TALKS \& POSTERS}
{\bf National \& International Conferences}
\begin{etaremune}

\item \href{https://indico.desy.de/event/27991/contributions/102162/}{International Cosmic Ray Conference 2021} \hfill 2021/07/20

\item \href{https://indico.ific.uv.es/event/3965/contributions/14963/}{Very Large Volume Neutrino Telescopes 2021 Plenary (invited)} \hfill 2021/05/19

\item \href{http://meetings.aps.org/Meeting/APR21/Session/T18}{APS April Meeting 2021} \hfill 2021/04/19

\item \href{https://sites.google.com/view/aapf2021/schedule}{19th Annual AAPF Symposium} \hfill 2021/02/09

\item \href{https://indico.fnal.gov/event/19348/contributions/186681/}{NEUTRINO 2020} \hfill 2020/06/21

\item \href{https://aapf-fellows.org/symposium/2020#BrianClark}{18th Annual AAPF Symposium at the 235th AAS Meeting, Honolulu HI.} \hfill 2020/01/04

\item \href{http://meetings.aps.org/Meeting/APR19/Session/R08.4}{APS April Meeting 2019, Denver CO} \hfill 2019/04/15
%\href{http://meetings.aps.org/Meeting/APR19/Session/R08.4}{{\em Searching for Neutrinos \& Cosmic Rays and Studying Antarctic ice with Askaryan Radio Array.}}

\item \href{http://meetings.aps.org/Meeting/APR18/Session/U17.7}{APS April Meeting 2018, Columbus OH}\hfill 2018/04/16
%\href{http://meetings.aps.org/Meeting/APR18/Session/U17.7}{{\em Directional Reconstruction as a Means of Lowering Thresholds for Point-Source Searches in the Askaryan Radio Array.}}

\item \href{http://indico.cern.ch/event/615891/contributions/2648790/}{TeV Particle Astrophysics, Columbus OH}\hfill 2017/08/11
%\href{http://indico.cern.ch/event/615891/contributions/2648790/}{{\em The Askaryan Radio Array: Current Status and Future Plans.} }

\item \href{http://meetings.aps.org/Meeting/APR17/Session/Y3.2}{APS April Meeting 2017, Washington DC} \hfill 2017/01/31
%\href{http://meetings.aps.org/Meeting/APR17/Session/Y3.2}{{\em Observation of Reconstructable Radio Waveforms from Solar Flares with Askaryan Radio Array.}}
\end{etaremune}
%\newpage
{\bf Colloquia, Seminars, and Other Talks}
\begin{etaremune}

\item University of Maryland High Energy Physics Seminar, College Park MD (invited) \hfill 2022/02/23


\item \href{https://drexel.edu/coas/news-events/events-calendar/details/?eid=35304&iid=93965}{Drexel Physics Colloquium}, Philadelphia PA (invited) \hfill 2022/02/17

\item \href{https://web.physics.udel.edu/events/colloquium/colloquium-brian-clark}{Univ. of Delaware Physics and Astronomy Colloquium}, Newark DE (invited) \hfill 2022/02/09

\item Univ. of Kansas Physics and Astronomy Colloquium, Lawrence KS (invited) \hfill 2021/11/22

\item MSU Astronomy Seminar, East Lansing MI \hfill 2019/10/23

\item OSU CCAPP Seminar, Columbus OH. \hfill 2019/07/16
%\textit{The Quest for Ultra-High Energy Neutrinos}

\item \href{http://meetings.aps.org/Meeting/OSF18/Session/A01.2}{Ohio Section of the APS Fall 2018 Meeting, Toledo OH.} \hfill 2018/09/29
%\href{http://meetings.aps.org/Meeting/OSF18/Session/A01.2}{\em Latest Results in the Search for Ultra-High Energy Neutrinos in the Askaryan Radio Array} 

%\item OSU Physics Summer Seminar Series, Columbus OH \hfill 2018/06/26
%{\em Ultra-High Energy Neutrino Astrophysics with Radio-Based Detectors.} 

\item \href{http://ccapp.osu.edu/pastseminars.html#past}{OSU CCAPP Seminar, Columbus OH} \hfill 2018/05/22
%\href{http://ccapp.osu.edu/pastseminars.html#past}{\textit{The Askaryan Radio Array: Detector Status and Prospects for Using Directional Reconstruction in Point-Source Searches.}}

\item Colloquium, College of Wooster Physics Department, Wooster OH (invited) \hfill 2016/10/04
%{\em Ultra-High Energy Neutrino Astrophysics with Radio Detectors.}

\item \href{http://ccapp.osu.edu/workshops/CHEAPR2016/workshop.html}{Computing in High Energy Astropart. Phys. Research 2016, Columbus OH.} \hfill 2016/05/26
%\href{http://ccapp.osu.edu/workshops/CHEAPR2016/workshop.html}{\em Machine Learning Prospects in Trigger Thresholds for High Energy Radio Neutrino Astronomy.}

\item OSU Physics Summer Seminar Series, Columbus OH \hfill 2016/04/23
%{\em Trigger Thresholds in High Energy Neutrino Astronomy.} 

\item \href{http://meetings.aps.org/Meeting/OSS16/Session/D3.6}{Ohio Section of the APS Spring 2016 Meeting, Dayton OH} \hfill 2016/04/09
%\href{http://meetings.aps.org/Meeting/OSS16/Session/D3.6}{\em Ultra-High Energy Neutrino Astrophysics with the Askaryan Radio Array (ARA).} 
\end{etaremune}


%\vspace{-0.35cm}

%----------------------------------------------------------------------------------------
%	OUTREACH
%----------------------------------------------------------------------------------------
\begin{rSection}{SERVICE and OUTREACH}

Early Career Scientists Representative, the IceCube Collaboration \hfill January 2021-Present \\
Physics Climate and Diversity Committee, OSU \hfill January 2017-May 2018\\
Officer, Physics Graduate Student Council, OSU \hfill October 2014-May 2017 \\
\vspace{-0.3cm}

Talk, \href{https://sciencefestival.msu.edu/schools/virtual-school-programs}{MSU Science Festival} \hfill April 2021 \\
Talk, \href{https://astronomy.osu.edu/events/making-space-all-chasing-ghost-particle}{Making Space for All} \hfill June 2020 \\
Talk, Astronomy on Tap Lansing \hfill October 2019, August 2021 \\
Coordinator for \href{u.osu.edu/aspire}{ASPIRE} Workshop for High School Women, OSU \hfill July 2015-June 2019\\
Volunteer Judge, Ohio State Science Day \hfill 2015-2019\\
Talk, Columbus Science Pub \hfill  May 2018\\
Talk, The Wellington School, Columbus, OH \hfill April 2018\\
%Volunteer Judge, OSU Denman Undergraduate Research Forum \hfill 2016\\
\end{rSection}
\vspace{-0.30cm}
\end{rSection}
\newpage

%----------------------------------------------------------------------------------------
%	MENTORSHIP
%----------------------------------------------------------------------------------------
\begin{rSection}{MENTORSHIP}
\begin{table}[h]
\begin{tabularx}{\textwidth}{l X}
 {\bf Graduate Students:}  & Lauren Ennesser, Hieu Le, Keith McBride, Andr\'es Medina, Jessie Micallef, Julie Rolla,  Jorge Torres-Espinosa \\
 ~ & ~ \\
{\bf Undergraduate Students:}  & Suren Gourapura, Emma Hettinger, Hannah Hassan, Jessica Kienbaum, Elizabeth Kowalczyk, Spoorthi Nagasmudram, Victoria Niu, Le Nguyen, Brandon Pries, Jude Rajasekera, Lucas Smith\\
~ & ~ \\
{\bf High School Students:} &  Addison Hartman, Natalie Keyes\\
\end{tabularx}
\end{table}

\end{rSection}
%%----------------------------------------------------------------------------------------
%%	REFERENCES
%%----------------------------------------------------------------------------------------
%\begin{rSection}{REFERENCES}
%\begin{tabular}{lr}
%% Referee 1
%\begin{minipage}[t]{2.5in}
%{\bf Amy Connolly}\\
%Professor of Physics\\
%The Ohio State University\\
%connolly@physics.osu.edu\\
%614-292-4368\\
%\end{minipage}
%&
%% Referee 2
%\begin{minipage}[t]{2.5in}
%{\bf Darren Grant}\\
%%(IceCube Spokesperson 2017-2021) \\
%Professor of Physics\\
%Michigan State University\\
%drg@msu.edu\\
%517-884-5567\\
%\end{minipage}
%\\
%\\ % Additional newline for spacing.
%% Referee 3
%\begin{minipage}[t]{2.5in}
%{\bf Shigeru Yoshida}\\
%Professor of Physics\\
%Chiba University\\
%syoshida@hepburn.s.chiba-u.ac.jp\\
%+81 (43) 290 - 3683 \\
%\end{minipage}
%&
%% Referee 2
%\begin{minipage}[t]{2.5in}
%{\bf Albrecht Karle}\\
%%(ARA PI, IceCube Assoc Director for Sci and Inst.) \\
%Professor of Physics\\
%University of Wisconsin - Madison\\
%albrecht.karle@icecube.wisc.edu\\
%608-262-3945\\
%\end{minipage}
%\\
%\\ % Additional newline for spacing.
%% Referee 3
%\begin{minipage}[t]{2.5in}
%{\bf Dave Besson}\\
%Professor of Physics and Astronomy\\
%The University of Kansas\\
%zedlam@ku.edu\\
%785-864-4741\\
%\end{minipage}
%&
%% Referee 2
%\begin{minipage}[t]{2.5in}
%{\bf Albrecht Karle}\\
%%(ARA PI, IceCube Assoc Director for Sci and Inst.) \\
%Professor of Physics\\
%University of Wisconsin - Madison\\
%albrecht.karle@icecube.wisc.edu\\
%608-262-3945\\
%\end{minipage}
%\end{tabular}
%\end{rSection}

\begin{rSection}{FULL PUBLICATION LIST}
\nocite{*}
%\setlength\bibitemsep{1.5\itemsep}
\setlength\bibitemsep{0.5\baselineskip}
\vspace{-0.5cm}
\printbibliography[title=\textcolor{white}{.}]
\end{rSection}

\end{document}
