%%%%%%%%%%%%%%%%%%%%%%%%%%%%%%%%%%%%%%%%%
% Medium Length Professional CV
% LaTeX Template
% Version 2.0 (8/5/13)
%
% This template has been downloaded from:
% http://www.LaTeXTemplates.com
%
% Original author:
% Trey Hunner (http://www.treyhunner.com/)
%
% Important note:
% This template requires the resume.cls file to be in the same directory as the
% .tex file. The resume.cls file provides the resume style used for structuring the
% document.
%
%%%%%%%%%%%%%%%%%%%%%%%%%%%%%%%%%%%%%%%%%

%----------------------------------------------------------------------------------------
%	PACKAGES AND OTHER DOCUMENT CONFIGURATIONS
%----------------------------------------------------------------------------------------
\documentclass{resume} % Use the custom resume.cls style
\usepackage{hyperref}
\usepackage{etaremune}
\usepackage{tabularx}
%\pagestyle{plain}
%\pagestyle{myheadings}
\usepackage{fancyhdr}
\pagestyle{fancy}
\fancyhf{}
\lhead{\vspace{0.3cm}\textit{Curriculum Vitae -- Brian A. Clark --} \thepage} 
\renewcommand{\headrulewidth}{0pt}
%\fancyhead{}
%\fancyfoot{}
%\fancyhead[LO,LE]{\textit{Curriculum Vitae --Brian A. Clark}}
%\fancyfoot[RO,RE]{\thepage}

\newenvironment{list1}{
  \begin{list}{\ding{113}}{%
      \setlength{\itemsep}{0in}
      \setlength{\parsep}{0in} \setlength{\parskip}{0in}
      \setlength{\topsep}{0in} \setlength{\partopsep}{0in} 
      \setlength{\leftmargin}{0.17in}}}{\end{list}}

\usepackage[left=0.75in,top=0.65in,right=0.75in,bottom=0.6in]{geometry} % Document margins
\name{Brian A. Clark} % Your name


%\address{B-1 \\ II , U.P. 208016} % Your address
%\address{123 Pleasant Lane \\ City, State 12345} % Your secondary addess (optional)
%\address{(+1)~7hh~9~8486 \\ x@ii.yzac.in} % Your phone number and email

\begin{document}

%----------------------------------------------------------------------------------------
%	CONTACT
%----------------------------------------------------------------------------------------

%191 W. Woodruff Ave  \hfill  {\it Phone:}  (614) 247-8268 \\
%Physics Research Building   \hfill {\it Email:}  clark.2668@osu.edu \\ 
%The Ohio State University   \hfill {\it Website:} \url{u.osu.edu/clark.2668} \\ 
%Columbus, OH  43210 USA   \hfill {\it OrcID / inSPIRE:} \,\,      \href{https://orcid.org/0000-0003-4089-2245}{0000-0003-4089-2245}  / \href{https://inspirehep.net/author/profile/Brian.A.Clark.1}{Brian.A.Clark.1} \\ 
\vspace{-1cm}
%\begin{rSection}{CONTACT}
\rule{\textwidth}{0.1cm} \\ \\
\begin{tabular}{@{}p{2in}p{4in}}
567 Wilson Rd Room 3243             & {\it Phone:}  (517) XXX-XXXX \\            
Biomed Phys Sci Building   & {\it Email:}  baclark@msu.edu
 \\         
Michigan State University & {\it Website:} \url{u.osu.edu/clark.2668} \\       
East Lansing, MI  48824 USA  & {\it OrcID / inSPIRE:} \,\,      \href{https://orcid.org/0000-0003-4089-2245}{0000-0003-4089-2245}  / \href{https://inspirehep.net/author/profile/Brian.A.Clark.1}{Brian.A.Clark.1}\\     
\end{tabular}
%\end{rSection}
%\vspace{0.25cm}

%----------------------------------------------------------------------------------------
%	RESEARCH INTERESTS
%---------------------------------------------------------------------------------------

\begin{rSection}{RESEARCH PROFILE}
National Science Foundation Astronomy and Astrophysics Postdoctoral Fellow working in experimental particle-astrophysics on the Askaryan Radio Array and IceCube experiments. Interested in high energy neutrino astronomy, specifically the construction, simulation, and data analysis of neutrino telescopes.
\end{rSection}
%\vspace{-0.1cm}

%----------------------------------------------------------------------------------------
%	EDUCATION
%----------------------------------------------------------------------------------------

\begin{rSection}{EDUCATION}
\textbf{Ph.D. in Physics, The Ohio State University}, Columbus, Ohio USA \hfill 2014-2019\\
\vspace*{-.15in}
\begin{list1}
\item[]Advisor: Prof. Amy Connolly
\end{list1}

\textbf{M.S. in Physics, The Ohio State University}, Columbus, Ohio USA \hfill 2014-2016

\textbf{B.A. in Physics, Washington University in St. Louis}, St. Louis, Missouri USA \hfill 2010-2014\\
\vspace*{-.15in}
\begin{list1}
\item[] \textit{Cum Laude}, Advisor: Prof. Henric Krawczynski
\end{list1}
\end{rSection}
%\vspace{-0.1cm}

%----------------------------------------------------------------------------------------
%	AWARDS
%----------------------------------------------------------------------------------------

\begin{rSection}{AWARDS}
National Science Foundation Astronomy and Astrophysics Postdoctoral Fellowship \hfill 2019-2021 \\
National Science Foundation Graduate Research Fellowship \hfill 2016-2019 \\
APS Division of Astrophysics Travel Award \hfill 2017, 2019 \\
Bunny and Thomas Clark Graduate Scholarship Honorable Mention \hfill 2019 \\
OSU Graduate Enrichment Fellowship \hfill 2014-2015 \\
WUSTL Undergraduate Physics Research Fellow \hfill Summer 2011 
\end{rSection}
%\vspace{-0.1cm}

%----------------------------------------------------------------------------------------
%	RESEARCH EXPERIENCE
%----------------------------------------------------------------------------------------

\begin{rSection}{RESEARCH EXPERIENCE}
{\bf Michigan State University}, East Lansing, MI USA \hfill {\bf August 2019 - present} \\
{\em Postdoctoral Fellow}%, High Energy Neutrino Astrophysics

{\bf The Ohio State University}, Columbus, OH USA \hfill {\bf August 2014 - July 2019} \\
{\em Ph.D. Student}%, Ultra-High Energy Neutrino Astrophysics
%\begin{itemize}
%\vspace*{-.05in}
%\item Developed frequency and time-series analysis techniques to analyze radio emission from solar flares in the ARA prototype station; this is the first extraterrestrial emission observed by the array.
%\item Implemente d filtering techniques to remove human-made noise from ARA data, and utilized them in a search for a diffuse flux of ultra-high energy neutrinos.
%\item Built and tested printed circuit boards for RF signal conditioning and power distribution, improving access to instrument dynamic range and operability in harsh environments.
%\item Led and directed the mechanical and electrical systems integration of three new neutrino detecting stations, including the management of a three person team of junior students.
%\item Deployed to Antarctica for five weeks to lead the commissioning and calibration of five neutrino detecting stations; performed rapid on site assessment of instrument performance.
%\end{itemize}

{\bf Washington University in St. Louis}, St. Louis, MO USA \hfill {\bf October 2012 - May 2014}\\
{\em Undergraduate Research Associate}%, X-Ray Astrophysics
%\begin{itemize}
%\vspace*{-.05in}
%\item Participated in X-Calibur collaboration to detect x-rays in the upper atmosphere, including fabrication of CCDs in a cleanroon environment.
%\item Wrote Monte Carlo simulations to explore Stokes parameters in x-ray astronomy by using methods of Bayesian confidence intervals.
%\end{itemize}
\end{rSection}

%----------------------------------------------------------------------------------------
%	PUBLICATIONS
%----------------------------------------------------------------------------------------
\begin{rSection}{PUBLICATIONS}
\begin{etaremune}%[leftmargin=0.64cm]
  \item ``NuRadioMC: Simulating the radio emission of neutrinos from interaction to detector" \\
 C. Glaser {\it et. al.} (incl. \textbf{B. A. Clark})\\    Submitted to Eur. Phys. J. C (2019). \href{https://arxiv.org/abs/1906.01670}{[arXiv:1906.01670]}
  \item ``Design and Performance of an Interferometric Trigger Array for Radio Detection of High-Energy Neutrinos" \\
 P. Allison {\it et. al.} for the ARA Collaboration (incl. \textbf{B. A. Clark}) \\    \href{https://doi.org/10.1016/j.nima.2019.01.067}{Nuclear Instruments and Methods A Vol 930 Pg 112-125 (2019).}  \href{https://arxiv.org/abs/1809.04573}{[arXiv:1809.04573]}
 \item ``Observation of Reconstructable Radio Emission Coincident with an X-Class Solar Flare in the Askaryan Radio Array Prototype Station." \\
 P. Allison {\it et. al.} for the ARA Collaboration (incl. \textbf{B. A. Clark} as corresponding author) \\
 Submitted to Astroparticle Physics (2018). \href{https://arxiv.org/abs/1807.03335}{[arXiv:1807.03335]}
  \item ``Measurement of the real dielectric permittivity $\epsilon_r$ of glacial ice." \\
 P. Allison {\it et. al.} for the ARA Collaboration (incl. \textbf{B. A. Clark}) \\
  \href{https://doi.org/10.1016/j.astropartphys.2019.01.004}{Astroparticle Physics Vol 108 Pg 63-73 (2019).} \href{https://arxiv.org/abs/1712.03301}{[arXiv:1712.03301]} 
   \item ``Analyzing the Data from X-ray Polarimeters with Stokes Parameters." \\
 F. Kislat,  \textbf{B. Clark}, M. Bielicke, H. Krawczynski.  \\
  \href{http://dx.doi.org/10.1016/j.astropartphys.2015.02.007}{Astroparticle Physics Vol 68 Pg 45-51 (2015).} \href{https://arxiv.org/abs/1409.6214}{[arXiv:1409.6214]} 
 \end{etaremune}
\end{rSection}

%----------------------------------------------------------------------------------------
%	SCIENTIFIC TALKS
%----------------------------------------------------------------------------------------

\begin{rSection}{SCIENTIFIC TALKS}
{\bf National \& International Conferences}
\begin{etaremune}
\item APS April Meeting, Denver CO. \hfill 2019/04/15 \\
\href{http://meetings.aps.org/Meeting/APR19/Session/R08.4}{{\em Searching for Neutrinos \& Cosmic Rays and Studying Antarctic ice with Askaryan Radio Array.}}

\item APS April Meeting, Columbus OH. \hfill 2018/04/16 \\
\href{http://meetings.aps.org/Meeting/APR18/Session/U17.7}{{\em Directional Reconstruction as a Means of Lowering Thresholds for Point-Source Searches in the Askaryan Radio Array.}}

\item TeV Particle Astrophysics, Columbus OH. \hfill 2017/08/11 \\
\href{http://indico.cern.ch/event/615891/contributions/2648790/}{{\em The Askaryan Radio Array: Current Status and Future Plans.} }

\item APS April Meeting, Washington DC. \hfill 2017/01/31 \\
\href{http://meetings.aps.org/Meeting/APR17/Session/Y3.2}{{\em Observation of Reconstructable Radio Waveforms from Solar Flares with Askaryan Radio Array.}}
\end{etaremune}

{\bf Colloquia, Seminars, and Other Talks}
\begin{etaremune}

\item OSU CCAPP Seminar, Columbus OH. \hfill 2019/07/16 \\
\textit{The Quest for Ultra-High Energy Neutrinos}

\item Ohio Section of the APS Fall 2018 Meeting, Toledo OH. \hfill 2018/09/29 \\
\href{http://meetings.aps.org/Meeting/OSF18/Session/A01.2}{\em Latest Results in the Search for Ultra-High Energy Neutrinos in the Askaryan Radio Array} 

\item OSU Physics Summer Seminar Series, Columbus OH. \hfill 2018/06/26 \\
{\em Ultra-High Energy Neutrino Astrophysics with Radio-Based Detectors.} 

\item OSU CCAPP Seminar, Columbus OH. \hfill 2018/05/22 \\
\href{http://ccapp.osu.edu/pastseminars.html#past}{\textit{The Askaryan Radio Array: Detector Status and Prospects for Using Directional Reconstruction in Point-Source Searches.}}

\item Colloquium, College of Wooster Physics Department, Wooster OH. \hfill 2016/10/04 \\
{\em Ultra-High Energy Neutrino Astrophysics with Radio Detectors.}

\item Computing in High Energy Astropart. Phys. Research 2016, Columbus OH. \hfill 2016/05/26 \\
\href{http://ccapp.osu.edu/workshops/CHEAPR2016/workshop.html}{\em Machine Learning Prospects in Trigger Thresholds for High Energy Radio Neutrino Astronomy.}

\item OSU Physics Summer Seminar Series, Columbus OH. \hfill 2016/04/23 \\
{\em Trigger Thresholds in High Energy Neutrino Astronomy.} 

\item Ohio Section of the APS Spring 2016 Meeting, Dayton OH. \hfill 2016/04/09 \\
\href{http://meetings.aps.org/Meeting/OSS16/Session/D3.6}{\em Ultra-High Energy Neutrino Astrophysics with the Askaryan Radio Array (ARA).} 
\end{etaremune}


\end{rSection}
%\vspace{-0.10cm}
%%\newpage
%%----------------------------------------------------------------------------------------
%%	RELEVANT SKILLS
%%----------------------------------------------------------------------------------------
%\begin{rSection}{RELEVANT SKILLS}
%\begin{tabular}{@{}l l l@{}}
%Programming/Software & & C++, C, Python, BASH, Energia, Code Composer Studio, PADS \\ 
%Mechanical/Electrical & & Surface mount soldering, power distribution, RF signal conditioning  \\ 
%\end{tabular}
%\end{rSection}
%----------------------------------------------------------------------------------------
%	TEACHING
%----------------------------------------------------------------------------------------
\newpage
\begin{rSection}{TEACHING}
{\bf The Ohio State University}, Columbus, OH 

{\em TA Training Facilitator, University Center for the Advancement Teaching} \hfill {\bf August 2016}
\begin{itemize}
\vspace*{-.05in}
\item Facilitated two-day ``introduction to teaching and learning`` workshop for 30 first-time Teaching Assistants across the University's 40 STEM science programs.
\item Built confidence in new TAs, guided development of teaching identities, addressed diversity in the classroom, and aided participant planning for long-term classroom success.
\end{itemize}
{\em Teaching Assistant--``Astronomy 1143: Stars, Galaxies, and Cosmology''} \hfill {\bf Spring 2016}
\begin{itemize}
\vspace*{-.05in}
\item Aided student learning by teaching review sessions and lecturing when lead faculty was absent for 80 student introductory survey course, open to students across the university
\item Moderated online forum, in collaboration with lead faculty, for students to exchange questions and clarify concepts.
\end{itemize}
{\em Teaching Assistant--``Physics 1251: E\&M, Optics, and Quantum Mechanics''} \hfill {\bf Fall 2015}
\begin{itemize}
\vspace*{-.05in}
\item Guided student learning in the recitation and laboratory context for four contact hours per week.
\item Facilitated quantitative laboratory experiments including team-based problem solving exercises.
\item Designed rubrics for fair, efficient, and consistent grading of quiz and examination instruments.
\end{itemize}
\end{rSection}
%\vspace{-0.35cm}

%----------------------------------------------------------------------------------------
%	OUTREACH
%----------------------------------------------------------------------------------------
\begin{rSection}{OUTREACH AND SERVICE}
Coordinator for \href{u.osu.edu/aspire}{ASPIRE} Workshop for High School Women, OSU \hfill July 2015-present\\
Physics Climate and Diversity Committee, OSU \hfill January 2017-May 2018\\
Volunteer Judge, Ohio State Science Day \hfill 2015-present\\
Talk, Columbus Science Pub \hfill  May 2018\\
Talk, The Wellington School, Columbus, OH \hfill April 2018\\
Officer, Physics Graduate Student Council, OSU \hfill October 2014-May 2017 \\
%Volunteer Judge, OSU Denman Undergraduate Research Forum \hfill 2016\\
\end{rSection}
\vspace{-0.30cm}

%----------------------------------------------------------------------------------------
%	MENTORSHIP
%----------------------------------------------------------------------------------------
\begin{rSection}{MENTORSHIP}
\begin{table}[h]
\begin{tabularx}{\textwidth}{l X}
 {\bf Graduate Students:}  & Lauren Ennesser, Keith McBride, Andr\'es Medina, Julie Rolla,  \hspace{1cm} Jorge Torres-Espinosa \\
 ~ & ~ \\
{\bf Undergraduate Students:}  & Ian Best, Eliot Ferstl, Suren Gourapura, Hannah Hassan, Scott Janse, Spoorthi Nagasmudram, Victoria Niu, Alex Patton, Jude Rajasekera, Cade Sbrocco, Lucas Smith, Jason Torok \\
~ & ~ \\
{\bf High School Students:} &  Addison Hartman, Natalie Keyes\\
\end{tabularx}
\end{table}
\end{rSection}

%----------------------------------------------------------------------------------------
%	REFERENCES
%----------------------------------------------------------------------------------------
\newpage
\begin{rSection}{REFERENCES}

\begin{tabular}{lr}
% Referee 1
\begin{minipage}[t]{2.5in}
{\bf Amy Connolly}\\
Professor of Physics\\
The Ohio State University\\
connolly@physics.osu.edu\\
614-292-4368\\
\end{minipage}
&
% Referee 2
\begin{minipage}[t]{2.5in}
{\bf Dave Besson}\\
Professor of Physics and Astronomy\\
The University of Kansas\\
zedlam@ku.edu\\
785-864-4741\\
\end{minipage}
\\
\\ % Additional newline for spacing.
% Referee 3
\begin{minipage}[t]{2.5in}
{\bf James Beatty}\\
Professor of Physics and Astronomy\\
The Ohio State University\\
beatty@mps.ohio-state.edu\\
614-247-8413\\
\end{minipage}
\end{tabular}

\end{rSection}

\end{document}
